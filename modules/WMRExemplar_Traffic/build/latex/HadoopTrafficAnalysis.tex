% Generated by Sphinx.
\def\sphinxdocclass{report}
\documentclass[letterpaper,10pt,openany,oneside]{sphinxmanual}
\usepackage[utf8]{inputenc}
\DeclareUnicodeCharacter{00A0}{\nobreakspace}
\usepackage[T1]{fontenc}
\usepackage[english]{babel}
\usepackage{times}
\usepackage[Bjarne]{fncychap}
\usepackage{longtable}
\usepackage{sphinx}
\usepackage{multirow}


\title{Hadoop Traffic Analysis}
\date{August 13, 2014}
\release{}
\author{CSInParallel Project}
\newcommand{\sphinxlogo}{}
\renewcommand{\releasename}{}
\makeindex

\makeatletter
\def\PYG@reset{\let\PYG@it=\relax \let\PYG@bf=\relax%
    \let\PYG@ul=\relax \let\PYG@tc=\relax%
    \let\PYG@bc=\relax \let\PYG@ff=\relax}
\def\PYG@tok#1{\csname PYG@tok@#1\endcsname}
\def\PYG@toks#1+{\ifx\relax#1\empty\else%
    \PYG@tok{#1}\expandafter\PYG@toks\fi}
\def\PYG@do#1{\PYG@bc{\PYG@tc{\PYG@ul{%
    \PYG@it{\PYG@bf{\PYG@ff{#1}}}}}}}
\def\PYG#1#2{\PYG@reset\PYG@toks#1+\relax+\PYG@do{#2}}

\expandafter\def\csname PYG@tok@gd\endcsname{\def\PYG@tc##1{\textcolor[rgb]{0.63,0.00,0.00}{##1}}}
\expandafter\def\csname PYG@tok@gu\endcsname{\let\PYG@bf=\textbf\def\PYG@tc##1{\textcolor[rgb]{0.50,0.00,0.50}{##1}}}
\expandafter\def\csname PYG@tok@gt\endcsname{\def\PYG@tc##1{\textcolor[rgb]{0.00,0.25,0.82}{##1}}}
\expandafter\def\csname PYG@tok@gs\endcsname{\let\PYG@bf=\textbf}
\expandafter\def\csname PYG@tok@gr\endcsname{\def\PYG@tc##1{\textcolor[rgb]{1.00,0.00,0.00}{##1}}}
\expandafter\def\csname PYG@tok@cm\endcsname{\let\PYG@it=\textit\def\PYG@tc##1{\textcolor[rgb]{0.25,0.50,0.56}{##1}}}
\expandafter\def\csname PYG@tok@vg\endcsname{\def\PYG@tc##1{\textcolor[rgb]{0.73,0.38,0.84}{##1}}}
\expandafter\def\csname PYG@tok@m\endcsname{\def\PYG@tc##1{\textcolor[rgb]{0.13,0.50,0.31}{##1}}}
\expandafter\def\csname PYG@tok@mh\endcsname{\def\PYG@tc##1{\textcolor[rgb]{0.13,0.50,0.31}{##1}}}
\expandafter\def\csname PYG@tok@cs\endcsname{\def\PYG@tc##1{\textcolor[rgb]{0.25,0.50,0.56}{##1}}\def\PYG@bc##1{\setlength{\fboxsep}{0pt}\colorbox[rgb]{1.00,0.94,0.94}{\strut ##1}}}
\expandafter\def\csname PYG@tok@ge\endcsname{\let\PYG@it=\textit}
\expandafter\def\csname PYG@tok@vc\endcsname{\def\PYG@tc##1{\textcolor[rgb]{0.73,0.38,0.84}{##1}}}
\expandafter\def\csname PYG@tok@il\endcsname{\def\PYG@tc##1{\textcolor[rgb]{0.13,0.50,0.31}{##1}}}
\expandafter\def\csname PYG@tok@go\endcsname{\def\PYG@tc##1{\textcolor[rgb]{0.19,0.19,0.19}{##1}}}
\expandafter\def\csname PYG@tok@cp\endcsname{\def\PYG@tc##1{\textcolor[rgb]{0.00,0.44,0.13}{##1}}}
\expandafter\def\csname PYG@tok@gi\endcsname{\def\PYG@tc##1{\textcolor[rgb]{0.00,0.63,0.00}{##1}}}
\expandafter\def\csname PYG@tok@gh\endcsname{\let\PYG@bf=\textbf\def\PYG@tc##1{\textcolor[rgb]{0.00,0.00,0.50}{##1}}}
\expandafter\def\csname PYG@tok@ni\endcsname{\let\PYG@bf=\textbf\def\PYG@tc##1{\textcolor[rgb]{0.84,0.33,0.22}{##1}}}
\expandafter\def\csname PYG@tok@nl\endcsname{\let\PYG@bf=\textbf\def\PYG@tc##1{\textcolor[rgb]{0.00,0.13,0.44}{##1}}}
\expandafter\def\csname PYG@tok@nn\endcsname{\let\PYG@bf=\textbf\def\PYG@tc##1{\textcolor[rgb]{0.05,0.52,0.71}{##1}}}
\expandafter\def\csname PYG@tok@no\endcsname{\def\PYG@tc##1{\textcolor[rgb]{0.38,0.68,0.84}{##1}}}
\expandafter\def\csname PYG@tok@na\endcsname{\def\PYG@tc##1{\textcolor[rgb]{0.25,0.44,0.63}{##1}}}
\expandafter\def\csname PYG@tok@nb\endcsname{\def\PYG@tc##1{\textcolor[rgb]{0.00,0.44,0.13}{##1}}}
\expandafter\def\csname PYG@tok@nc\endcsname{\let\PYG@bf=\textbf\def\PYG@tc##1{\textcolor[rgb]{0.05,0.52,0.71}{##1}}}
\expandafter\def\csname PYG@tok@nd\endcsname{\let\PYG@bf=\textbf\def\PYG@tc##1{\textcolor[rgb]{0.33,0.33,0.33}{##1}}}
\expandafter\def\csname PYG@tok@ne\endcsname{\def\PYG@tc##1{\textcolor[rgb]{0.00,0.44,0.13}{##1}}}
\expandafter\def\csname PYG@tok@nf\endcsname{\def\PYG@tc##1{\textcolor[rgb]{0.02,0.16,0.49}{##1}}}
\expandafter\def\csname PYG@tok@si\endcsname{\let\PYG@it=\textit\def\PYG@tc##1{\textcolor[rgb]{0.44,0.63,0.82}{##1}}}
\expandafter\def\csname PYG@tok@s2\endcsname{\def\PYG@tc##1{\textcolor[rgb]{0.25,0.44,0.63}{##1}}}
\expandafter\def\csname PYG@tok@vi\endcsname{\def\PYG@tc##1{\textcolor[rgb]{0.73,0.38,0.84}{##1}}}
\expandafter\def\csname PYG@tok@nt\endcsname{\let\PYG@bf=\textbf\def\PYG@tc##1{\textcolor[rgb]{0.02,0.16,0.45}{##1}}}
\expandafter\def\csname PYG@tok@nv\endcsname{\def\PYG@tc##1{\textcolor[rgb]{0.73,0.38,0.84}{##1}}}
\expandafter\def\csname PYG@tok@s1\endcsname{\def\PYG@tc##1{\textcolor[rgb]{0.25,0.44,0.63}{##1}}}
\expandafter\def\csname PYG@tok@gp\endcsname{\let\PYG@bf=\textbf\def\PYG@tc##1{\textcolor[rgb]{0.78,0.36,0.04}{##1}}}
\expandafter\def\csname PYG@tok@sh\endcsname{\def\PYG@tc##1{\textcolor[rgb]{0.25,0.44,0.63}{##1}}}
\expandafter\def\csname PYG@tok@ow\endcsname{\let\PYG@bf=\textbf\def\PYG@tc##1{\textcolor[rgb]{0.00,0.44,0.13}{##1}}}
\expandafter\def\csname PYG@tok@sx\endcsname{\def\PYG@tc##1{\textcolor[rgb]{0.78,0.36,0.04}{##1}}}
\expandafter\def\csname PYG@tok@bp\endcsname{\def\PYG@tc##1{\textcolor[rgb]{0.00,0.44,0.13}{##1}}}
\expandafter\def\csname PYG@tok@c1\endcsname{\let\PYG@it=\textit\def\PYG@tc##1{\textcolor[rgb]{0.25,0.50,0.56}{##1}}}
\expandafter\def\csname PYG@tok@kc\endcsname{\let\PYG@bf=\textbf\def\PYG@tc##1{\textcolor[rgb]{0.00,0.44,0.13}{##1}}}
\expandafter\def\csname PYG@tok@c\endcsname{\let\PYG@it=\textit\def\PYG@tc##1{\textcolor[rgb]{0.25,0.50,0.56}{##1}}}
\expandafter\def\csname PYG@tok@mf\endcsname{\def\PYG@tc##1{\textcolor[rgb]{0.13,0.50,0.31}{##1}}}
\expandafter\def\csname PYG@tok@err\endcsname{\def\PYG@bc##1{\setlength{\fboxsep}{0pt}\fcolorbox[rgb]{1.00,0.00,0.00}{1,1,1}{\strut ##1}}}
\expandafter\def\csname PYG@tok@kd\endcsname{\let\PYG@bf=\textbf\def\PYG@tc##1{\textcolor[rgb]{0.00,0.44,0.13}{##1}}}
\expandafter\def\csname PYG@tok@ss\endcsname{\def\PYG@tc##1{\textcolor[rgb]{0.32,0.47,0.09}{##1}}}
\expandafter\def\csname PYG@tok@sr\endcsname{\def\PYG@tc##1{\textcolor[rgb]{0.14,0.33,0.53}{##1}}}
\expandafter\def\csname PYG@tok@mo\endcsname{\def\PYG@tc##1{\textcolor[rgb]{0.13,0.50,0.31}{##1}}}
\expandafter\def\csname PYG@tok@mi\endcsname{\def\PYG@tc##1{\textcolor[rgb]{0.13,0.50,0.31}{##1}}}
\expandafter\def\csname PYG@tok@kn\endcsname{\let\PYG@bf=\textbf\def\PYG@tc##1{\textcolor[rgb]{0.00,0.44,0.13}{##1}}}
\expandafter\def\csname PYG@tok@o\endcsname{\def\PYG@tc##1{\textcolor[rgb]{0.40,0.40,0.40}{##1}}}
\expandafter\def\csname PYG@tok@kr\endcsname{\let\PYG@bf=\textbf\def\PYG@tc##1{\textcolor[rgb]{0.00,0.44,0.13}{##1}}}
\expandafter\def\csname PYG@tok@s\endcsname{\def\PYG@tc##1{\textcolor[rgb]{0.25,0.44,0.63}{##1}}}
\expandafter\def\csname PYG@tok@kp\endcsname{\def\PYG@tc##1{\textcolor[rgb]{0.00,0.44,0.13}{##1}}}
\expandafter\def\csname PYG@tok@w\endcsname{\def\PYG@tc##1{\textcolor[rgb]{0.73,0.73,0.73}{##1}}}
\expandafter\def\csname PYG@tok@kt\endcsname{\def\PYG@tc##1{\textcolor[rgb]{0.56,0.13,0.00}{##1}}}
\expandafter\def\csname PYG@tok@sc\endcsname{\def\PYG@tc##1{\textcolor[rgb]{0.25,0.44,0.63}{##1}}}
\expandafter\def\csname PYG@tok@sb\endcsname{\def\PYG@tc##1{\textcolor[rgb]{0.25,0.44,0.63}{##1}}}
\expandafter\def\csname PYG@tok@k\endcsname{\let\PYG@bf=\textbf\def\PYG@tc##1{\textcolor[rgb]{0.00,0.44,0.13}{##1}}}
\expandafter\def\csname PYG@tok@se\endcsname{\let\PYG@bf=\textbf\def\PYG@tc##1{\textcolor[rgb]{0.25,0.44,0.63}{##1}}}
\expandafter\def\csname PYG@tok@sd\endcsname{\let\PYG@it=\textit\def\PYG@tc##1{\textcolor[rgb]{0.25,0.44,0.63}{##1}}}

\def\PYGZbs{\char`\\}
\def\PYGZus{\char`\_}
\def\PYGZob{\char`\{}
\def\PYGZcb{\char`\}}
\def\PYGZca{\char`\^}
\def\PYGZam{\char`\&}
\def\PYGZlt{\char`\<}
\def\PYGZgt{\char`\>}
\def\PYGZsh{\char`\#}
\def\PYGZpc{\char`\%}
\def\PYGZdl{\char`\$}
\def\PYGZti{\char`\~}
% for compatibility with earlier versions
\def\PYGZat{@}
\def\PYGZlb{[}
\def\PYGZrb{]}
\makeatother

\begin{document}

\maketitle
\tableofcontents
\phantomsection\label{index::doc}


This module was created for CSInParallel by Jeffrey Lyman in 2014
(\href{mailto:JLyman@macalester.edu}{JLyman@macalester.edu})

The purpose of this module is to teach students how to analyze
datasets distributed over multiple files using the
Hadoop framework. It is assumed that students are already
familiar with the basics of hadoop and CSInParallel's Web Map
Reduce hadoop interface.

The exercises in this module use a dataset from the UK
department of Transportation that contains detailed records of
traffic accidents split into three separate files.

The dataset can be obtained from
\href{http://academictorrents.com/details/c7d2d7a91ae3fd0256dd2ba2d7344960cb3c4dbb}{Academic Torrents}.
More about the source of this public data from the United Kingdom can be found on
\href{http://en.wikipedia.org/wiki/Reported\_Road\_Casualties\_Great\_Britain}{its Wikipedia page}.


\chapter{Contents:}
\label{index:hadoop-traffic-analysis}\label{index:contents}

\section{Introduction to the Dataset}
\label{0-TrafficDataset/Introduction:introduction-to-the-dataset}\label{0-TrafficDataset/Introduction::doc}

\subsection{The Data}
\label{0-TrafficDataset/Introduction:the-data}
The UK department of Transportation keeps detailed records
of all traffic incidents. Fortunately for us they made this
data available to the public in the form of three csv files
that contain information about the accidents, casualties,
and vehicles involved.
\setbox0\vbox{
\begin{minipage}{0.95\linewidth}
\textbf{System-dependent Alert}

\medskip


The path of the dataset shown below may not be the same on your WMR system.
It is correct for this WMR server:

selkie.macalester.edu/wmr
\end{minipage}}
\begin{center}\setlength{\fboxsep}{5pt}\shadowbox{\box0}\end{center}

These files are located on wmr in the \code{/shared/traffic}
folder and are named \code{Accidents7409.csv,  Casualty7409.csv and
Vehicles7409.csv} respectively.


\subsection{Working with the Data}
\label{0-TrafficDataset/Introduction:working-with-the-data}
Each line in the files contains several fields separated by
commas, to access these values, it is necessary to call
\code{key.split(',')} (or the equivalent in whatever language
you're using) to get an array of values. If you want, you
can turn these values into an object, however it's faster
to simply refer to them by their index

\begin{longtable}{|l|l|l|l|}
\hline
\textbf{
index
} & \textbf{
Accidents7409.csv
} & \textbf{
Casualty7409.csv
} & \textbf{
Vehicles7409.csv
}\\\hline
\endfirsthead

\multicolumn{4}{c}%
{{\bfseries \tablename\ \thetable{} -- continued from previous page}} \\
\hline
\textbf{
index
} & \textbf{
Accidents7409.csv
} & \textbf{
Casualty7409.csv
} & \textbf{
Vehicles7409.csv
}\\\hline
\endhead

\hline \multicolumn{4}{|r|}{{Continued on next page}} \\ \hline
\endfoot

\hline
\endlastfoot


0
 & 
Accident Index
 & 
Accident Index
 & 
Accident Index
\\\hline

1
 & 
Location Easting OSGR
 & 
Vehicle Reference
 & 
Vehicle Reference
\\\hline

2
 & 
Location Northing OSGR
 & 
Casualty Reference
 & 
Vehicle Type
\\\hline

3
 & 
Longitude
 & 
Casualty Class
 & 
Towing/Articulation
\\\hline

4
 & 
Latitude
 & 
Sex of Casualty
 & 
Vehicle Maneuver
\\\hline

5
 & 
Police Force
 & 
Age Band of Casualty
 & 
Vehicle Location Restricted Lane
\\\hline

6
 & 
Accident Severity
 & 
Casualty Severity
 & 
Junction Location
\\\hline

7
 & 
Number of Vehicles
 & 
Pedestrian Location
 & 
Skidding/Overturning
\\\hline

8
 & 
Number of Casualties
 & 
Pedestrian Movement
 & 
Hit Object in Driveway
\\\hline

9
 & 
Date
 & 
Car Passenger
 & 
Vehicle Leaving Driveway
\\\hline

10
 & 
Day of Week
 & 
Bus/Coach Passenger
 & 
Hit Object off Driveway
\\\hline

11
 & 
Time
 & 
Pedestrian Road  Maintenance Worker
 & 
1st Point of Impact
\\\hline

12
 & 
Local Authority (District)
 & 
Casualty Type
 & 
Was Vehicle Left Hand Drive
\\\hline

13
 & 
Local Authority (Highway)
 & 
Casualty Home Area Type
 & 
Journey Purpose of Driver
\\\hline

14
 & 
1st Road Class
 &  & 
Sex of Driver
\\\hline

15
 & 
1st Road Number
 &  & 
Age Band of Driver
\\\hline

16
 & 
Road Type
 &  & 
Engine Capacity
\\\hline

17
 & 
Speed Limit
 &  & 
Propulsion Code
\\\hline

18
 & 
Junction Detail
 &  & 
Age of Vehicle
\\\hline

19
 & 
Junction Control
 &  & 
Driver IMD Decile
\\\hline

20
 & 
2nd Road Class
 &  & 
Driver Home Area Type
\\\hline

21
 & 
2nd Road Number
 &  & \\\hline

22
 & 
Pedestrian Crossing Human Control
 &  & \\\hline

23
 & 
Pedestrian Crossing Physical Facilities
 &  & \\\hline

24
 & 
Light Conditions
 &  & \\\hline

25
 & 
Weather Conditions
 &  & \\\hline

26
 & 
Road Surface Conditions
 &  & \\\hline

27
 & 
Special Conditions
 &  & \\\hline

28
 & 
Carriage Hazards
 &  & \\\hline

29
 & 
Urban or Rural Area
 &  & \\\hline

30
 & 
Did Police Officer Attend Scene
 &  & \\\hline

31
 & 
LSOA of Accident Location
 &  & \\\hline
\end{longtable}


Most of the values are determined by special codes which
which can be found in the pages of
\code{this spreadsheet}


\subsection{Example Job}
\label{0-TrafficDataset/Introduction:example-job}
Let's use what we've learned to answer a quick question.
Between 1974 and 2004 were there more casualties per
incident in rural or urban accidents?

Our \code{mapper} will need to emit a key
that represents whether the accident was rural or urban and the number
of casualties as the value.

Our \code{reducer} will need to sum the
casualties for each type of accident and divide them by the total number
of accidents.

Given that the code that tells whether a crash was urban or
rural is stored at index 29 of the accident csv and the
number of casualties is stored at index 8 our code looks like
this:

\begin{Verbatim}[commandchars=\\\{\},numbers=left,firstnumber=1,stepnumber=1]
\PYG{k}{def} \PYG{n+nf}{mapper}\PYG{p}{(}\PYG{n}{key}\PYG{p}{,} \PYG{n}{value}\PYG{p}{)}\PYG{p}{:}
  \PYG{n}{data} \PYG{o}{=} \PYG{n}{key}\PYG{o}{.}\PYG{n}{split}\PYG{p}{(}\PYG{l+s}{'}\PYG{l+s}{,}\PYG{l+s}{'}\PYG{p}{)}
  \PYG{n}{casualties} \PYG{o}{=} \PYG{n}{data}\PYG{p}{[}\PYG{l+m+mi}{8}\PYG{p}{]}
  \PYG{n}{urbanOrRural} \PYG{o}{=} \PYG{n}{data}\PYG{p}{[}\PYG{l+m+mi}{29}\PYG{p}{]}
  \PYG{n}{Wmr}\PYG{o}{.}\PYG{n}{emit}\PYG{p}{(}\PYG{n}{urbanOrRural}\PYG{p}{,} \PYG{n}{casualties}\PYG{p}{)}

\PYG{k}{def} \PYG{n+nf}{reducer}\PYG{p}{(}\PYG{n}{key}\PYG{p}{,} \PYG{n}{values}\PYG{p}{)}\PYG{p}{:}
  \PYG{n}{count} \PYG{o}{=} \PYG{l+m+mi}{0}
  \PYG{n}{total} \PYG{o}{=} \PYG{l+m+mi}{0}
  \PYG{k}{for} \PYG{n}{value} \PYG{o+ow}{in} \PYG{n}{values}\PYG{p}{:}
    \PYG{n}{total} \PYG{o}{+}\PYG{o}{=} \PYG{n+nb}{int}\PYG{p}{(}\PYG{n}{value}\PYG{p}{)}
    \PYG{n}{count} \PYG{o}{+}\PYG{o}{=} \PYG{l+m+mi}{1}
  \PYG{n}{Wmr}\PYG{o}{.}\PYG{n}{emit}\PYG{p}{(}\PYG{n}{key}\PYG{p}{,} \PYG{n}{total} \PYG{o}{/} \PYG{n}{count}\PYG{p}{)}
\end{Verbatim}

\begin{notice}{note}{Note:}
Does this reducer look familiar?
\end{notice}

Run this job on wmr using cluster path
\code{/shared/traffic/Accidents7904.csv} You should get the following
output:

\begin{tabulary}{\linewidth}{|L|L|}
\hline

1
 & 
1.2805146224316546
\\\hline

2
 & 
1.5105844913989401
\\\hline

3
 & 
1.4071045576407506
\\\hline

-1
 & 
1.3062582787269292
\\\hline
\end{tabulary}


A quick glance at the spreadsheet reveals that 1 stands for
Urban, 2 for rural, and 3 for unallocated. -1 means that neither
was reported. It appears that on average rural accidents tend
to involve more casualties.


\section{Working with Multiple files}
\label{1-MulitipleFiles/Multiples::doc}\label{1-MulitipleFiles/Multiples:working-with-multiple-files}
The sample question from the last section was fairly simple
to answer because all of the data could be found in one
file. However data is often split between files, making it
harder to process.

Take this question for instance: are taxis more likely to
get into crashes on the weekend?


\subsection{Taxi Crashes}
\label{1-MulitipleFiles/Multiples:taxi-crashes}
To answer this question we will need to access the day of
week data at accidents{[}10{]} and the vehicle type data at
vehicles{[}2{]} (codes 8 and 108 represent taxis). However those
two bits of data are in two separate files so we'll need some
way to cross reference them. We'll do that with the accident
index stored at accidents{[}0{]} and vehicles{[}0{]}

This also means that we'll need to access multiple files during
a single job. Luckily WMR makes this easy for us. If we enter
a folder into the cluster path, it will use all the files in
that folder has input.

However we still need to be able to tell if a mapper key came
from the accidents file or the vehicles file. We can do this
by looking at the length of the data list. The Vehicles file
has 21 pieces of information while the Accidents file has 32.
Armed with this information we can write a mapper and a reducer
that will filter out accidents based on whether they involved
a taxi. Run \code{this code} using Cluster Path \code{/shared/traffic}

\begin{Verbatim}[commandchars=\\\{\},numbers=left,firstnumber=1,stepnumber=1]
\PYG{k}{def} \PYG{n+nf}{mapper}\PYG{p}{(}\PYG{n}{key}\PYG{p}{,} \PYG{n}{value}\PYG{p}{)}\PYG{p}{:}
    \PYG{n}{data} \PYG{o}{=} \PYG{n}{key}\PYG{o}{.}\PYG{n}{split}\PYG{p}{(}\PYG{l+s}{'}\PYG{l+s}{,}\PYG{l+s}{'}\PYG{p}{)}
    \PYG{k}{if} \PYG{n+nb}{len}\PYG{p}{(}\PYG{n}{data}\PYG{p}{)} \PYG{o}{==} \PYG{l+m+mi}{21}\PYG{p}{:}                 \PYG{c}{\PYGZsh{}vehicle data}
        \PYG{k}{if} \PYG{n}{data}\PYG{p}{[}\PYG{l+m+mi}{2}\PYG{p}{]} \PYG{o+ow}{in} \PYG{p}{(}\PYG{l+s}{'}\PYG{l+s}{8}\PYG{l+s}{'}\PYG{p}{,} \PYG{l+s}{'}\PYG{l+s}{108}\PYG{l+s}{'}\PYG{p}{)}\PYG{p}{:}     \PYG{c}{\PYGZsh{}codes for taxis}
            \PYG{n}{Wmr}\PYG{o}{.}\PYG{n}{emit}\PYG{p}{(}\PYG{n}{data}\PYG{p}{[}\PYG{l+m+mi}{0}\PYG{p}{]}\PYG{p}{,} \PYG{l+s}{"}\PYG{l+s}{taxi}\PYG{l+s}{"}\PYG{p}{)}
    \PYG{k}{elif} \PYG{n+nb}{len}\PYG{p}{(}\PYG{n}{data}\PYG{p}{)} \PYG{o}{==} \PYG{l+m+mi}{32}\PYG{p}{:}               \PYG{c}{\PYGZsh{}accident data}
        \PYG{n}{Wmr}\PYG{o}{.}\PYG{n}{emit}\PYG{p}{(}\PYG{n}{data}\PYG{p}{[}\PYG{l+m+mi}{0}\PYG{p}{]}\PYG{p}{,} \PYG{n}{data}\PYG{p}{[}\PYG{l+m+mi}{10}\PYG{p}{]}\PYG{p}{)}
\end{Verbatim}

This mapper checks to see whether input came from accident data
or vehicle data. Then, if it was accident data, it emits the
day of the week that the accident occurred on. If it came from
the vehicles data then it emits a message if a vehicle involved
was a taxi.

Our \code{reducer} takes that output and emits a list of accident
indices and the day of the week that they occurred on.

\begin{Verbatim}[commandchars=\\\{\},numbers=left,firstnumber=1,stepnumber=1]
\PYG{k}{def} \PYG{n+nf}{reducer}\PYG{p}{(}\PYG{n}{key}\PYG{p}{,} \PYG{n}{values}\PYG{p}{)}\PYG{p}{:}
    \PYG{n}{isTaxi} \PYG{o}{=} \PYG{n+nb+bp}{False}
    \PYG{n}{dayOfWeek} \PYG{o}{=} \PYG{l+s}{"}\PYG{l+s}{"}
    \PYG{k}{for} \PYG{n}{value} \PYG{o+ow}{in} \PYG{n}{values}\PYG{p}{:}
        \PYG{k}{if} \PYG{n}{value} \PYG{o}{==} \PYG{l+s}{"}\PYG{l+s}{taxi}\PYG{l+s}{"}\PYG{p}{:}
            \PYG{n}{isTaxi} \PYG{o}{=} \PYG{n+nb+bp}{True}
        \PYG{k}{else}\PYG{p}{:}
            \PYG{n}{dayOfWeek} \PYG{o}{=} \PYG{n}{value}
    \PYG{n}{Wmr}\PYG{o}{.}\PYG{n}{emit}\PYG{p}{(}\PYG{n}{dayOfWeek}\PYG{p}{,} \PYG{n}{key}\PYG{p}{)}
\end{Verbatim}

This works because only one day of week value is emitted per
accident index and while there can be more than one taxi
involved in a given crash.

But we're not done yet. We simply have list of crashes and
a list of the days on which they occurred. We still need to
count them.

We can this by using the output of the last job to run a new
job. Just hit the use output button at the top
or bottom of the page.

Our mapper will receive days of the week as keys and ones as the values.
We just need to feed these straight into a
\code{counting reducer}
by using what's known as the \code{identity mapper}
our code is as follows:

\begin{Verbatim}[commandchars=\\\{\},numbers=left,firstnumber=1,stepnumber=1]
\PYG{k}{def} \PYG{n+nf}{mapper}\PYG{p}{(}\PYG{n}{key}\PYG{p}{,} \PYG{n}{value}\PYG{p}{)}\PYG{p}{:}
    \PYG{n}{Wmr}\PYG{o}{.}\PYG{n}{emit}\PYG{p}{(}\PYG{n}{key}\PYG{p}{,} \PYG{n}{value}\PYG{p}{)}
\end{Verbatim}

\begin{Verbatim}[commandchars=\\\{\},numbers=left,firstnumber=1,stepnumber=1]
\PYG{k}{def} \PYG{n+nf}{reducer}\PYG{p}{(}\PYG{n}{key}\PYG{p}{,} \PYG{n}{values}\PYG{p}{)}
    \PYG{n}{count} \PYG{o}{=} \PYG{l+m+mi}{0}
    \PYG{k}{for} \PYG{n}{value} \PYG{o+ow}{in} \PYG{n}{values}\PYG{p}{:}
        \PYG{n}{count} \PYG{o}{+}\PYG{o}{=} \PYG{n+nb}{int}\PYG{p}{(}\PYG{n}{value}\PYG{p}{)}
    \PYG{n}{emit}\PYG{p}{(}\PYG{n}{key}\PYG{p}{,} \PYG{n}{count}\PYG{p}{)}
\end{Verbatim}

After submitting the job on WMR we get the following output:

\begin{tabulary}{\linewidth}{|L|L|}
\hline

1
 & 
693847
\\\hline

2
 & 
873422
\\\hline

3
 & 
877086
\\\hline

4
 & 
890605
\\\hline

5
 & 
934161
\\\hline

6
 & 
1058859
\\\hline

7
 & 
896218
\\\hline
\end{tabulary}


Code 1 is Sunday, code 2 is Monday etc. So it looks like
Taxis get into the most accidents on Fridays, a fairly high
number on Saturdays, but very few on Sundays.


\subsection{Challenges}
\label{1-MulitipleFiles/Multiples:challenges}
Use the techniques you've learned to answer the following
questions, or come up with your own:
\begin{itemize}
\item {} 
Are male drivers more likely to injure other males? You
will need the following fields: Sex of the driver -
Vehicles{[}14{]}, Sex of casualty - Casualties{[}4{]} in both
cases 1 is male 2 is female 3 is unknown and -1 is
missing data.

\item {} 
What is the average severity of a crash in which at
least one vehicle overturned? If vehicles{[}7{]} = 2, 5, or
4 the vehicle overturned. The severity of an accident
is Accidents{[}6{]} and ranges from 1-3, 1 being the most
serious.

\item {} 
Are trucks more deadly than vans?

\item {} 
Create a graph showing the number of traffic accidents
at each hour of the day. If you're feeling adventurous
separate it out by day and hour.

\item {} 
Devise some of your own questions to ask of this data.

\end{itemize}



\renewcommand{\indexname}{Index}
\printindex
\end{document}
